\textbf{THE IMPACT OF ROBOT TUTOR SOCIAL BEHAVIOUR ON CHILDREN}\newline
\textbf{JAMES KENNEDY}

Robotic technologies possess great potential to enter our daily lives because they have the ability to interact with our world. But our world is inherently social. Whilst humans often have a natural understanding of this complex environment, it is much more challenging for robots. The field of social Human-Robot Interaction (HRI) seeks to endow robots with the characteristics and behaviours that would allow for intuitive multimodal interaction. Education is a social process and previous research has found strong links between the social behaviour of teachers and student learning. This therefore presents a promising  application opportunity for social human-robot interaction.

The thesis presented here is that a robot with \gls{tailored} social behaviour will positively influence the outcomes of tutoring interactions with children and consequently lead to an increase in child \gls{learning} when compared to a robot without this social behaviour. It has long been established that one-to-one tutoring provides a more effective means of learning than the current typical school classroom model (one teacher to many students). Schools increasingly supplement their teaching with technology such as tablets and laptops to offer this personalised experience, but a growing body of evidence suggests that robots lead to greater learning than other media. It is posited that this is due to the increased social presence of a robot. This work adds to the evidence that robots hold a social advantage over other technological media, and that this indeed leads to increased learning.

In addition, the work here contributes to existing knowledge by seeking to expand our understanding of how to manipulate robot social behaviour in educational interactions such that the behaviour is tailored for this purpose. To achieve this, a means of characterising social behaviour is required, as is a means of measuring the success of the behaviour for the interaction. To characterise the social behaviour of the robot, the concept of \textit{immediacy} is taken from the human-human literature and validated for use in HRI. Greater use of immediacy behaviours is also tied to increased cognitive learning gains in humans. This can be used to predict the same effect for the use of social behaviour by a robot, with learning providing an objective measure of success for the robot behaviour given the education application.

It is found here that when implemented on a robot in tutoring scenarios, greater use of immediacy behaviours generally does tend to lead to increased learning, but a complex picture emerges. Merely the addition of more social behaviour is insufficient to increase learning; it is found that a balance should be struck between the addition of social cues, and the congruency of these cues.