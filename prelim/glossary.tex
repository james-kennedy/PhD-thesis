\usepackage[acronym,toc]{glossaries}

\makeglossaries
 
\newglossaryentry{learning}
{
name={learning},
description={For a full discussion of definitions of learning, please see Section \ref{sec:background-learning}. Throughout the thesis, learning will often refer specifically to cognitive learning gains that require the application of newly acquired knowledge to a novel problem, i.e., more than mere recall of information}
}

\newglossaryentry{tailored}
{
name={tailored},
description={Used in reference to the thesis of this work, `tailored' describes the intentional design of social behaviours such that they are adapted specifically for the application domain of educational interactions}
}

\newglossaryentry{immediacy}
{
name={immediacy},
description={Describes both a set of concepts for characterising social behaviour, and a scale used to measure this construct. Please see \gls{verbalimm} and \gls{nonverbalimm} for further details}
}

\newglossaryentry{verbalimm}
{
name={verbal immediacy},
description={Addresses the verbal aspects of the immediacy construct, or can refer to the scale for these aspects. These include a variety of personalisation elements, such as revealing personal information in an interaction, as well as encouraging thoughts from students. A human-based verbal immediacy questionnaire can be seen in \citet{gorham1988relationship}}
}

\newglossaryentry{nonverbalimm}
{
name={nonverbal immediacy},
description={Addresses the nonverbal aspects of the immediacy construct, or can refer to the scale for these aspects. These include a number of nonverbal social cues, such as: gestures, gaze, vocal prosody, touch, facial expressions, and proximity. Please see Section \ref{sec:method-nvi} for an introduction to the scale used to measure nonverbal immediacy}
}
 
\newacronym{chri}{cHRI}{Child-Robot Interaction} 

\newacronym{hri}{HRI}{Human-Robot Interaction}

\newacronym{hhi}{HHI}{Human-Human Interaction}

\newacronym{hci}{HCI}{Human-Computer Interaction}

\newacronym{ci}{CI}{Confidence Interval}

\newacronym{rniq}{RNIQ}{Robot Nonverbal Immediacy Questionnaire}

\newacronym{riq}{RIQ}{Robot Immediacy Questionnaire}

\newacronym{cniq}{CNIQ}{Child Nonverbal Immediacy Questionnaire}

\newacronym{its}{ITS}{Intelligent Tutoring Systems}

\newacronym{aied}{AIED}{Artificial Intelligence in Education}

\newacronym{alize}{ALIZ-E}{Adaptive Strategies for Sustainable Long-Term Social Interaction (European FP7 project)}

\newacronym{dream}{DREAM}{Development of Robot-Enhanced Therapy for Children with Autism Spectrum Disorders (European FP7 project)}

\newacronym{l2tor}{L2TOR}{Second Language Tutoring using Social Robots (European Horizon 2020 project)}

\newacronym{woz}{WoZ}{Wizard-of-Oz}

\newacronym{nvi}{NVI}{Nonverbal Immediacy}