\chapter{Contribution and Conclusion} \label{chap:conclusion}
This chapter seeks to provide an overview of the findings and topics covered in this thesis. The contributions to the field of social \acrshort{hri} are outlined and summarised. Following this, a conclusion is provided to briefly encapsulate the primary outcome of this work.

%%%%%%%%%%%%%%%%%%%%%%%%%%%%%%%%%%%%%%%%%%%%%%%%%%%%%%%%%%%%%%%%%%%%%%%%%%
\section{Summary}\label{sec:conc-summary}
The background to this thesis introduced a variety of work that suggested a difference in the way that humans perceive and interact with a physically present, real robot when compared to a virtual, on-screen robot. Some studies had shown that the embodiment of the real robot led to differences in compliance with robot instructions \citep{bainbridge2008effect}, or in task performance \citep{leyzberg2012physical}. Some work had suggested that differences were also present in \gls{learning} outcomes, but \gls{learning} measures were unclear or strict experimental controls were not in place \citep{han2008comparative}. As such, this was highlighted as an area that this research could make a contribution to.

Chapter \ref{chap:embodiment} sought to explore the impact of robot embodiment on child behaviour and \gls{learning}. A real and virtual robot were compared in a sorting task with children. There were no \gls{learning} differences between the conditions, however this may have been due in part to complications with measuring \gls{learning} with a biased dataset. Nonetheless, behavioural differences were found between embodiment conditions: children gazed more towards the real robot than the virtual one. This is an encouraging first step towards \gls{learning}, given the social basis of \gls{learning} \citep{kuhl2007speech} and the important role of gaze in establishing attention \citep{wu2010social}.

These findings were furthered in Chapter \ref{chap:socasoc} where a control condition with the robot removed was used in a prime number sorting task. It was found that when a robot was present, statistically significant \gls{learning} occurred, but when the same information came from just the touchscreen, the \gls{learning} was not significant. This observation provides support for the utility of using social robots in child education, with clear benefits in terms of cognitive \gls{learning} outcomes. Other contemporary work reports similar \gls{learning} advantages when robots are used in addition to other media. For example, \cite{alemi2014employing} show that when using a robot in addition to a human teacher, children learn significantly more. Agreement between these studies, despite differing child ages and learning materials, contribute to a growing body of evidence demonstrating the advantages that real robots can confer in education.

However, Chapter \ref{chap:socasoc} also revealed surprising findings, where a robot behaviour considered to have lower social skills, led to more \gls{learning} than another with higher social skills. This was in line with the background to this research, which explored the impact of various robot social behaviours on human \gls{learning}, revealing a complex and apparently inconsistent picture. It was suggested that a measure to characterise social behaviour would be beneficial, as this would aid researchers in the comparison of results in the context of different social behaviours. \textit{Immediacy} was identified as an appropriate measure due to its extensive use in human-human interaction research and ties with \gls{learning} gains. This was adapted and validated by the author for use with robots and with children in Chapter \ref{chap:validation}.

The two components of immediacy: \gls{nonverbalimm} and \gls{verbalimm} were then explored in Chapters \ref{chap:nviexperiment} and \ref{chap:verbal}, respectively. Robot social behaviours were designed (and verified) to be of high or low nonverbal (Chapter \ref{chap:nviexperiment}), or verbal (Chapter \ref{chap:verbal}) immediacy. It was found that higher \gls{nonverbalimm} led to increased cognitive \gls{learning}, as predicated by the human-human literature, but that verbal immediacy did not affect child \gls{learning}.

Chapter \ref{chap:behavemodel} brought together many of the results from the studies undertaken as part of this research, along with a human condition for comparison. These were tied together through the use of \gls{nonverbalimm} ratings from both children and adults (through online crowdsourcing of observed interaction clips). Limitations in the utility of \gls{nonverbalimm} for characterising interactions (as opposed to social cues) were highlighted and the data collected was used to propose a model for the design of social behaviour for maximising \gls{learning} gains. Limitations of the research were then raised and discussed; many of these led to suggestions for future work, which are expanded on in Section \ref{sec:conc-futurework} below.

The synthesis of the experiments in Chapter \ref{chap:behavemodel}, as well as the discussions that have taken place alongside each experiment in earlier chapters, show that there is data to support the thesis: a robot with \gls{tailored} social behaviour will lead to greater child \gls{learning} than a robot without this social behaviour. However, it was also shown that the precise implementation of this social behaviour is not always straightforward. It is necessary to factor in not just the social cues that the robot exhibits, but how congruent social cues are with one another, within the restrictions that certain robotic platforms can impose.

%%%%%%%%%%%%%%%%%%%%%%%%%%%%%%%%%%%%%%%%%%%%%%%%%%%%%%%%%%%%%%%%%%%%%%%%%%
\section{Contributions}\label{sec:conc-contribution}
This section will revisit the contributions outlined in the introduction (Chapter \ref{chap:intro}), with further expansion and explanation. The main contributions of this thesis are as follows:

\begin{itemize}

	\item \textbf{Further evidence for the advantage of physical robots for child social responses}, particularly in gaze towards the robot, when compared to virtual agents (Chapter \ref{chap:embodiment} and \citealp{kennedy2015comparing}). It appears that children respond to a physical robot with greater gaze than they do with a virtual, on-screen robot. This difference in social response is encouraging for the benefits of using a physically present robot for \acrshort{chri}.
	
	\item \textbf{Further evidence for the advantage of robots on \gls{learning} outcomes} with findings to show that children learn more when a robot is present than when only a touchscreen is used, regardless of having the same lesson content (Chapter \ref{chap:socasoc} and \citealp{kennedy2015robot}). This is a promising finding which supports the use of robots for tutoring children.
	
	\item \textbf{Adaptation and development of scales for use with children for characterising verbal and nonverbal social behaviour of humans and robots} (Chapters \ref{chap:method}, \ref{chap:validation}, \ref{chap:verbal}, and \citealp{kennedy2016social}). These questionnaires are based on the short-form \gls{nonverbalimm} questionnaire \citep{richmond2003development} and the verbal immediacy questionnaire \citep{gorham1988relationship}. The language used was adapted to be suitable for children, and with robots. They can be seen in the appendices: Child Nonverbal Immediacy Questionnaire (\acrshort{cniq}), the Robot Nonverbal Immediacy Questionnaire (\acrshort{rniq}), and the Robot Immediacy Questionnaire (\acrshort{riq}).
	
	\item \textbf{Validation of the \acrshort{cniq} and \acrshort{rniq} with children and adults for humans and robots} (Chapter \ref{chap:validation}). The questionnaires were validated with a large sample of children and adults to verify that the questionnaires were suitable for use with children, and also that the concepts mapped to \acrshort{hri}. It was found that the questionnaires were reasonably reliable and \gls{learning} effects predicted from the \acrshort{hhi} literature were observed, providing further reassurance of the applicability of immediacy as a metric for \acrshort{chri}.
		
	\item \textbf{Findings to show that a robot with higher \gls{nonverbalimm} leads to greater child \gls{learning}} (Chapter \ref{chap:nviexperiment} and \citealp{kennedy2015higher}). When an increased number of \gls{nonverbalimm} cues are utilised, such as gaze, gestures, and so on, child \gls{learning} increases. This demonstrates the value of improving robot social behaviour for child \gls{learning} outcomes.
	
	\item In short-term interactions, \textbf{higher \gls{verbalimm} of a robot does not seem to improve the \gls{learning} of children}, nor the retention of this \gls{learning} (Chapter \ref{chap:verbal} and \citealp{kennedy2016social}). Verbal immediacy centres around personalisation and aspects of `friendliness' in dialogue, such as using the child's name, soliciting opinions, revealing personal information, and more besides. This does not appear to impact on \gls{learning} in short interactions, although it is suggested that it may manifest into greater differences in the longer-term.
	
	\item \textbf{A proposed model of the relationship between robot social behaviour and child \gls{learning}}. This model incorporates the quantity and congruence of social cues, where a combination of a high number of social cues with high congruency will lead to maximal possible \gls{learning} (Chapter \ref{chap:behavemodel}). This model is derived from the data collected throughout the experimental work undertaken here, and can be used to produce further hypotheses for future work.
	
\end{itemize}

%%%%%%%%%%%%%%%%%%%%%%%%%%%%%%%%%%%%%%%%%%%%%%%%%%%%%%%%%%%%%%%%%%%%%%%%%%
\section{Conclusion}\label{sec:conc-conc}
The thesis presented here is that a robot with \gls{tailored} social behaviour will positively influence the outcomes of tutoring interactions with children and consequently lead to an increase in child \gls{learning} when compared to a robot without this social behaviour. The work undertaken adds to the evidence that robots hold a social advantage over other technological media, and that this indeed leads to increased \gls{learning}. Whilst evidence gathered through a series of experiments supports the thesis, it is also found that care must be taken when seeking to tailor social behaviour with the aim of improving interaction outcomes (as measured through \gls{learning}). By characterising social behaviour using immediacy, it was shown that a greater use of immediacy behaviours generally does tend to lead to increased \gls{learning}, but a complex picture emerges. If more social cues are used, but they are not congruent with one another, then this can negatively affect the \gls{learning} outcome of the interaction. Merely the addition of more social behaviour is insufficient to increase \gls{learning}; it is found that a balance should be struck between the addition of social cues, and the congruency of these cues.